%%% symbols, notations, etc.
\usepackage{physics,braket,bm,amssymb} % physics and math
\renewcommand{\t}{\text} % text in math mode
\newcommand{\f}[2]{\dfrac{#1}{#2}} % shorthand for fractions
\newcommand{\p}[1]{\left(#1\right)} % parenthesis
\renewcommand{\sp}[1]{\left[#1\right]} % square parenthesis
\renewcommand{\set}[1]{\left\{#1\right\}} % curly parenthesis
\newcommand{\bk}{\Braket} % shorthand for braket notation
\renewcommand{\d}{\text{d}} % for infinitesimals
\renewcommand{\v}{\bm} % bold vectors
\newcommand{\uv}[1]{\bm{\hat{#1}}} % unit vectors
\newcommand{\av}{\vec} % arrow vectors
\renewcommand{\c}{\cdot} % inner product
\newcommand{\del}{\nabla} % del operator
\newcommand{\w}{\wedge} % wedge product

\usepackage{dsfont} % for identity operator
\newcommand{\1}{\mathds{1}}

%%% figures
\usepackage{graphicx} % for figures
\usepackage{float} % for [H] placement option
\usepackage{grffile} % help latex properly identify figure extensions
\graphicspath{{./figures/}} % set path for all graphics
\usepackage[caption=false]{subfig} % subfigures (via \subfloat[]{})
\newcommand{\sref}[1]{\protect\subref{#1}} % for referencing subfigures
\usepackage{multirow} % multirow entries in tables
\usepackage{footnote} % footnotes in floats
\usepackage[font=small,labelfont=bf]{caption} % caption text options


%% enumeration / lists
\usepackage[inline]{enumitem} % in-line lists and \setlist{} (below)
\setlist[enumerate,1]{label={(\roman*)}} % default in-line numbering
\setlist{nolistsep} % more compact spacing between environments
\setlist[itemize]{leftmargin=*} % nice margins for itemize ...
\setlist[enumerate]{leftmargin=*} % ... and enumerate environments


%%% miscellaneous markup
% normalem included to prevent underlining titles in the bibliography
\usepackage[normalem]{ulem} % for strikeout text
\usepackage{color,soul} % text color and other editing options
% \ul{underline}, \st{strikethrough}, and \hl{highlight}


% color definitions
\usepackage{xcolor}
\definecolor{lightblue}{RGB}{31,119,180}
\definecolor{orange}{RGB}{255,127,14}
\definecolor{green}{RGB}{44,160,44}


%%% "double-hat" on symbols
\usepackage{amsmath,accents}
\newlength{\dhatheight}
\newcommand{\hatt}[1]{%
  \settoheight{\dhatheight}{\ensuremath{\hat{#1}}}%
  \addtolength{\dhatheight}{-0.35ex}%
  {\Hat{\vphantom{\rule{1pt}{\dhatheight}}%
      \smash{\Hat{#1}}}}
}
\newcommand{\uv}[1]{\bm{{\hat{#1}}}} % unit vectors


%%%%%%%%%%%%%%%%%%%%%%%%%%%%%%%%%%%%%%%%%%%%%%%%%%%%%%%%%%%%%%%%%%%%%%
%%% tensor network drawing tools
%%% taken in part from arxiv.org/abs/1603.03039

\usepackage{xcolor}
\definecolor{tensorblue}{rgb}{0.8,0.8,1}
\definecolor{tensorred}{rgb}{1,0.5,0.5}
\definecolor{tensorpurp}{rgb}{1,0.5,1}

\usepackage{tikz}
% \usetikzlibrary{external}
% \tikzexternalize
% \tikzsetexternalprefix{./tikz/}

\newcommand{\diagram}[2][0.5]{
  ~\begin{tikzpicture}
    [scale=#1, every node/.style = {sloped,allow upside down},
    baseline = {([yshift=+0ex]current bounding box.center)},
    rounded corners=2pt]
    #2
  \end{tikzpicture}~}

\tikzset{tens/.style={fill=tensorblue}}
\tikzset{diag/.style={fill=green!50!black!50}}
\tikzset{isom/.style={fill=orange!30}}
\tikzset{proj/.style={fill=tensorred}}

\tikzset{tengrey/.style={fill=black!20}}
\tikzset{tenpurp/.style={fill=tensorpurp}}

\usetikzlibrary{calc}
\newcommand{\wire}[4][]{
  \draw[#1] (#3,#2) -- (#4,#2)
}
\newcommand{\vwire}[4][]{
  \draw[#1] (#2,#3) -- (#2,#4)
}
\newcommand{\dwire}[5][]{
  \draw[#1] (#4,#2) -- (#5,#3)
}
\newcommand{\rect}[6][tens]{
  \draw[#1] (#4,#2) rectangle (#5,#3);
  \draw ($ (#4,#2) !.5! (#5,#3) $) node {$#6$}
}
\renewcommand{\circ}[4][tens]{
  \draw[#1] (#3,#2) circle (0.5) node {$#4$};
}

\newcommand{\Circ}[5][tens]{
  \draw[#1] (#3,#2) circle (#4) node {$#5$};
}

\renewcommand{\dot}[2]{
  \node at (#2,#1) [circle,fill,inner sep=1.5pt]{};
}
\renewcommand{\cross}[3][.25]{
  \draw (#3,#2) circle (#1) node {};
  \draw ($ (#3,#2) - (#1,0) $) -- ($ (#3,#2) + (#1,0) $) node{};
  \draw ($ (#3,#2) - (0,#1) $) -- ($ (#3,#2) + (0,#1) $) node{};
}
\newcommand{\cnot}[3]{
  \dot{#2}{#1};
  \vwire{#1}{#2}{#3};
  \cross{#3}{#1};
}

%%%%%%%%%%%%%%%%%%%%%%%%%%%%%%%%%%%%%%%%%%%%%%%%%%%%%%%%%%%%%%%%%%%%%%



%%% for scattering diagrams
\usepackage{tikz}
\tikzset{
  baseline = (current bounding box.center)
}
\usepackage{tikz-feynman}
\tikzfeynmanset{
  compat = 1.1.0,
  every diagram = {small}
}
\newcommand{\shrink}[1]{\scalebox{0.8}{#1}} % for smaller diagrams


%%% footnote fixes
\setlength{\footnotesep}{12pt} % fix line separation in footnotes
% letters instead of numbers for marking footnotes
\renewcommand*{\thefootnote}{\alph{footnote}}
% place after \begin{document}
\count\footins = 1000 % allows for longer footnotes

% modify indentation of footnotes
\makeatletter
\renewcommand<>\beamer@framefootnotetext[1]{%
  \global\setbox\beamer@footins\vbox{%
    \hsize\framewidth
    \textwidth\hsize
    \columnwidth\hsize
    \unvbox\beamer@footins
    \reset@font\footnotesize
    \@parboxrestore
    \protected@edef\@currentlabel
         {\csname p@footnote\endcsname\@thefnmark}%
    \color@begingroup
      \uncover#2{\@makefntext{%
        \rule\z@\footnotesep\ignorespaces\parbox[t]{.9\textwidth}{#1\@finalstrut\strutbox}\vskip1sp}}%
    \color@endgroup}%
}
\makeatother

% proper coloring inside math environment
\makeatletter
\def\mathcolor#1#{\@mathcolor{#1}}
\def\@mathcolor#1#2#3{
  \protect\leavevmode
  \begingroup
    \color#1{#2}#3
  \endgroup
}
\makeatother



%%% bibliography
\usepackage[sort&compress,numbers]{natbib} % bibliography options
\bibliographystyle{apsrev4-1}
\setlength{\bibsep}{0.0pt} % no spacing between bibliography entries
\usepackage{doi} % dereference DOI links
\usepackage[english]{babel} % allow editing bibliography title
\addto\captionsenglish{ % set bibliography title
  \renewcommand\refname{\normalsize\bf References}}


%%% importing code from files
\usepackage{listings}



%%% one column header for two column document
\newcommand{\head}[1]{
\twocolumn[\begin{@twocolumnfalse}
\vspace{-5mm}
\begin{center} #1 \end{center}
\end{@twocolumnfalse}]}



%%% indented paragraph
\newenvironment{indpar}
{\begin{list}{}
	{\setlength{\leftmargin}{5mm}\setlength{\rightmargin}{5mm}}
 	\item[]
}
{\end{list}}



%%% cyrillic script
\usepackage[utf8]{inputenc}
\usepackage[russian,english]{babel}
\usepackage[T2A,OT1]{fontenc}
\renewcommand{\cyr}[1]{{\fontencoding{T2A}\selectfont \em{#1}}}



% energy level diagram:
\begin{tikzpicture}[
  scale=0.5,
  level/.style={thick},
  virtual/.style={thick,densely dashed},
  trans/.style={thick,<->,shorten >=2pt,shorten <=2pt,>=stealth}
  ]
  % draw the energy levels
  \draw[level] (2cm,-2em) -- (0cm,-2em) node[left] {$\ket{a}$};
  \draw[level] (2cm,11em) -- (4cm,11em) node[midway,above] {$\ket{b}$};
  \draw[level] (4cm,-11em) -- (6cm,-11em) node[right] {$\ket{c}$};
  % draw the virtual levels
  \draw[virtual] (2cm,8em) -- (4cm,8em) node[midway,above] {$\Delta_1$};
  \draw[virtual] (4cm,-8em) -- (6cm,-8em) node[midway,below] {$\Delta_2$};
  \draw[virtual] (2cm,-5em) -- (0cm,-5em) node[midway,above] {$\Delta_3$};
  % draw the transitions
  \draw[trans] (1cm,-2em) -- (2.5cm,8em) node[midway,left] {$\Omega_1$};
  \draw[trans] (3.5cm,8em) -- (5cm,-8em) node[midway,right] {$\Omega_2$};
\end{tikzpicture}